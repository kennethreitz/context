% example showing how two trees can be merged ``by hand''
% that is, you can take two trees generated by desc-tex2,
% and put them aside, with a perfecr alignment, in that you create
% invisible nodes. Just run this file through TeX to see what I mean.
%
% This was made after a suggestion by Michael P. Gerlek
%
% D. Roegel, 16 January 1995
%
\input pstricks
\input poster
\input drsetup
\Poster[hcenter=true,%
        vcenter=true,%
        clip=pstricks,%
        cropwidth=.4pt,paperwidth=210mm,paperheight=297mm]
\vbox{\setbox0=\vbox{%    Because \Poster processes in horizontal mode,
%   but the tree macros are in vertical mode.
\tree{unframed;norules}%
    % Beginning of first tree
    \subtree{framed;rules:right}%
    node B1
    \subtree{normal}%
    long node B2
    \subtree{normal}%
    node B3
    \endsubtree
    \endsubtree
    \endsubtree
    % End of first tree
\subtree{unframed;norules}% once removed
    % Beginning of second tree
    \subtree{framed;rules:right}%
    node C
    \subtree{normal}%
    C1
    \endsubtree
    \subtree{normal}%
    C2
    \subtree{normal}%
    C2.1
    \endsubtree
    \endsubtree
    \endsubtree
    % End of second tree
\endsubtree
\endtree
}% End of \vbox
\copy0
\vskip1cm
\noindent\rlap{\hbox to\wd0{\hss%\credits
                            }}}
\endPoster
\end
